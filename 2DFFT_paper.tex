\documentclass[twocolumn]{aastex61}
\pdfoutput=1 
\usepackage{upgreek}

\usepackage{amsmath}
%\usepackage{graphicx}

%\usepackage{deluxetable}
%\usepackage{morefloats}

%\usepackage{amssymb}
%\usepackage{amsmath}
%\usepackage{epstopdf}
%\usepackage{color}

%\usepackage{mathrsfs}

\def \frb {FRB Search Algorithm}
\def \halpha {\ensuremath{\mathrm{H\alpha}}}
\def \hbeta {\ensuremath{\mathrm{H\beta}}}


\newcommand{\be}{\begin{eqnarray}}
\newcommand{\ee}{\end{eqnarray}}
\newcommand{\etal}{et al.}


\newcommand{\Halpha}{\rm H\alpha}
\newcommand{\EM}{{\rm EM}}
\newcommand{\DM}{{\rm DM}}
\newcommand{\SM}{{\rm SM}}
\newcommand{\SFR}{\ensuremath{\mathrm{SFR}}}

\newcommand{\DMhatzero}{\rm \widehat{\DM}}
\newcommand{\DMhatFRB}{\rm \widehat{\DM}({\rm FRB})}

\newcommand{\Ss}{S({\rm\halpha})_{\rm s}}
\newcommand{\EMs}{{\rm EM}_{\rm s}}
\newcommand{\EMsHalpha}{{\rm EM}(\halpha)_{\rm s}}
\newcommand{\EMsHbeta}{{\rm EM}(\hbeta)_{\rm s}}
\newcommand{\DMsHalpha}{{\rm DM}(\halpha)_{\rm s}}
\newcommand{\DMs}{\widehat{\rm DM}_{\rm s}}


\newcommand{\EMDM}{{\rm EM_{\rm DM}}}
\newcommand{\EMSM}{{\rm EM_{\rm SM}}}

\turnoffedit

\begin{document}

%\setlength{\fboxsep}{0pt}%
%\setlength{\fboxrule}{0.5pt}%

%\received{\today}
\revised{\today}
%\accepted{\today}
%\submitjournal{ApJ}

\shorttitle{2D-FFT search Algorithm }
\shortauthors{C.H.Niu~et al.}

\title{Using 2D-FFT to search Fast Radio Burst\footnote{peterniu@nao.cas.cn}}


\correspondingauthor{Xuelei~Chen}
\email{}

\author{C.H. Niu}
\affil{Central China Normal University \\
Luoyu Road,Wuhan, China}
\affil{National Astronomy Observatory , Chinese Academy of Sciences \\
Datun(A) Road,No.30,Beijing, China}
\affil{University of California Berkeley, Campbell Hall 339, Berkeley CA 94720}

\author{Y.C. Li}
\affil{National Astronomy Observatory , Chinese Academy of Sciences \\
Datun(A) Road,No.30,Beijing, China}

\author{F.Q. Wu}
\affil{National Astronomy Observatory , Chinese Academy of Sciences \\
Datun(A) Road,No.30,Beijing, China}

\author{Ue-li Pen}
\affil{CITA}
\affil{National Astronomy Observatory , Chinese Academy of Sciences \\
Datun(A) Road,No.30,Beijing, China}

\author{X.L. Chen}
\affil{National Astronomy Observatory , Chinese Academy of Sciences \\
Datun(A) Road,No.30,Beijing, China}







\begin{abstract}

Fast Radio Burst have been found from pulsar data for many years. There are several FRB search algorithm like tree algorithm, FDMT et. Here we proposed a different FRB searching algorithm which basically trace a curve in frequency-time image. This algorithm is mainly realized by two dimensional Fast Fourier Transform. We take a 2D FFT on the $f^{-2}(t)$ data map, Then trace the signal along the angle of straight line. In this searching method, it's easier to remove RFI in large scale and will bring a speed up benefit in well-developed 2D FFT library both in CPU and GPU code. \\


Fast Radio Burst is a high energy radio signal found in the Universe. The first one is found by Lorimer Duncn in 2007, now people always call it as Lormeter burst. Like Pulsar, It’s a wide band radio sginal, when it go through the inter stellar or inter galaxy medium, the higher frequency will go faster than lower frequency. When Signal go through dense of ISM  The origin of FRB is still unclear, there are lots of theories trying to describe what FRB is.  \\






% We discuss the implications of the first FRB host being a low-metallicity dwarf galaxy for source models, future searches for FRB hosts, and FRB rates.

\end{abstract}

\keywords{2DFFT ,FAST radio burst }

\section{Introduction}
 \label{sec:intro}
Fast radio bursts (FRBs) are bright ($\sim$Jy) and short ($\sim$ms) bursts of radio emission that have dispersion measures (DMs) in excess of the line of sight DM contribution expected from the electron distribution of our Galaxy. To date 18 FRBs have been reported
%\footnote{\url{http://www.astronomy.swin.edu.au/pulsar/frbcat/}} \citep{pbj+16}
 --- most of them detected at the Parkes telescope \citep{lbm+07,tsb+13,bb14,kskl12,rsj15,pbb+15,kjb+16,cpk+16,rsb+16} and one each at the Arecibo \citep{sch+14} and Green Bank telescopes \citep{mls+15}. 

A plethora of source models have been proposed to explain the properties of FRBs \citep[see e.g.][for a brief review]{katz16}. According to the models, the excess DM for FRBs may be intrinsic to the source, placing it within the Galaxy; it may arise mostly from the intergalactic medium, placing a source of FRBs at cosmological distances ($z\sim0.2-1$) or it may arise from the host galaxy, placing a source of FRBs at extragalactic, but not necessarily cosmological, distances ($\sim100$\,Mpc).

Since the only evidence to claim an extragalactic origin for FRBs has been the anomalously high DM, some models also attempted to explain the excess DM as a part of the model, thus allowing FRBs to be Galactic. All FRBs observed to date have been detected with single dish radio telescopes, for which the localization is of order arcminutes, insufficient to obtain an unambiguous association with any object. To date, no independent information about their redshift, environment, and source could be obtained due to the lack of an accurate localization of FRBs. \citet{kjb+16} attempted to identify the host of FRB\,150418 on the basis of a fading radio source in the field that was localized to a $z=0.492$ galaxy. However, later work identified the radio source as a variable active galactic nucleus (AGN) that may not be related to the source \citep{wb16,bbt+16,gmg+16,jkb+16}.

%\subsection{Interferometric Localization of \frb}
 Repeated radio bursts were observed from the location of the Arecibo-detected \frb\ \citep{ssh+16a,ssh+16b}, with the same DM as the first detection, indicating a common source. As discussed by \citet{ssh+16a}, it is unclear whether the repetition makes \frb\ unique among known FRBs, or whether radio telescopes other than Arecibo lack the sensitivity to readily detect repeat bursts from other known FRBs.

\citet{clw+16} used the Karl G. Jansky Very Large Array (VLA) to directly localize the repeated bursts from \frb\ with 100-mas precision and reported an unresolved, persistent radio source and an extended optical counterpart at the location with a chance coincidence probability of $\approx 3\times10^{-4}$ --- the first unambiguous identification of multi-wavelength counterparts to FRBs. Independently, \citet{mph+16} used the European VLBI Network (EVN) to localize the bursts and the persistent source and showed that both are co-located within $\sim12$ milliarcseconds.

Here we report a new algorithm to search FRBs . 

%In Section~\ref{sec:obs}, we discuss the observations and data analysis. In Section~\ref{sec:results}, we identify the counterpart as the host galaxy of the FRB and present its redshift and spectral characteristics. In Section~\ref{sec:discussion}, we discuss the physical properties of the environment and the implications for source models for FRBs.

%Constraining the multitude of FRB models is complicated by the difficulty in localizing FRBs, as all have been found in beams of single dish radio telescopes, which are several arcminutes in diameter. Realtime detection and rapid multi-wavelength follow-up have so far been unsuccessful \citep{pbb+15,kjb+16}.

%\textit{Paragraph on repeating FRB 121102 and VLA localization.}


\section{Basics of Incoherent Dedispersion}
\label{sec:obs}

The dispersion of the electromagnaetic wave pulse cause a delay in arrival time at frequency $\nu$ compared with thre reference frequency $\nu_0$, which is given by :
\begin{equation}
\Delta t (\nu) = -D(\nu^{-2} -\nu^{-2}_{0})
\end{equation}
where D is the dispersion measure. Thus , We may model a burst with a very short intrinsic width as :
\begin{equation}
I(t,\nu)=I_0(\nu)\delta_D(t-t_s -\frac{D}{\nu^2})
\end{equation}
Where $\delta D$ is the Dirac delta function, $t_s$ marks the signal starting time for infinitely high frequency. If the bandwidth is small, we can approximate
\begin{equation*}
\frac{D}{\nu^{2}}\approx\frac{D}{\nu_0^2}(1-2\frac{\nu-\nu_0}{\nu_0})
\end{equation*}
denote $\Delta \nu \equiv \nu - \nu_0$, and assume that the spectrum is not too steep such that within the observing band the signal is constant, then
\begin{equation}
\begin{aligned}
I(t,\nu) & \approx I_0\delta_D(t-t_s-\frac{D}{\nu_0^2}(1-2\frac{\Delta\nu}{\nu_0}) ) \\
		 & = I_0\delta_D(t -t_0 +\frac{2D}{\nu_0^3}\Delta\nu)
\end{aligned} 
\end{equation}
where $t_0$ is the arrival time of the signal at thre reference frequency $\nu_0$. \\
	Now consider an integral of this signal between frequency $\nu_1 $ and $\nu_2$ , the signal strength would be
\begin{equation}
s = \int d \nu \int dtI(t,\nu)=(\nu_2 -\nu_1)I_0 = I_0B
\end{equation}
Where $B =\nu_2 -\nu_1$ is the bandwidth. Now consider the noise. Suppose the data is digitized with time interval $\delta t$ and frequency channel bandwidth $\delta \nu$. For the incoherent dedispersion, the signal within each time interval and frequency channel is 
\begin{equation}
I_n = \frac{2k T_{sys}}{A_{\rm eff} \sqrt{\delta \nu \delta t}} 
\end{equation}
Suppose we are observing between $\nu_1$,$\nu_2$ with a total of $N_\nu$ channels, and processing a time interval $T = N_t \delta t$ where $T \geq \Delta t(\nu_1) -\Delta t(\nu_2)$, i.e. the whole of the dispersed signal is within the data frame. \\
	For incoherent dedispersion , in the absence of the pulse signal, the whole read out of the data frame is given by 
	\begin{equation}
	\begin{aligned}
	n &=  \int d\nu\int dt I_n = \frac{2kT_{sys}}{A_{\rm eff}}\frac{(\nu_2 -\nu_1) T}{\sqrt{\delta \nu \delta t}} \\
	  &=  \frac{2kT_{sys}}{A_{\rm eff}} B^{1/2} T^{1/2}N_{\nu}^{1/2}N_t^{1/2}
	\end{aligned}
	\end{equation}
So the raw signal to noise ratio is given 
\begin{equation}
{\rm SNR}_{raw} = \frac{I_0 A_{\rm eff}}{2kT_{sys}}\left(\frac{B}{N_\nu N_t T}\right)^{1/2}
\end{equation}

	 In a perfect incoherent dedispersion , we sum up all the signal, which is still given by $s$. However, we compare it with the noise in the same dedispersion $\nu -t $ track, not the whole data frame. The noise along the same track is given by 
	 \begin{equation}
	 n = \int d\nu \int dt I_n \delta _D(t- t_0 + \frac{2D}{\nu_0 ^3}\Delta \nu)=BI_n
	 \end{equation}
Then 
\begin{equation}
{\rm SNR} _{opt} = \frac{I_0}{I_N}=\frac{I_0 A_{\rm eff}}{2kT_{sys}} \left( \frac{BT}{N_{\nu} N_t} \right)^{1/2}
\end{equation}

Now consider a pulse of finite width. We replace the Dirac $\delta$ function by a Gaussian function with the same normalization
\begin{equation}
\delta _D(t-t') \to g(t-t') \equiv \frac{1}{(2\pi)^{1/2} \sigma} \exp[-\frac{(t-t')^2}{2\sigma^2_t}]
\end{equation}
If the pulse intrinsic width $\sigma > \delta t$, then in a dedispersion along the track only the part of the signal within one time bin would be included, which gives
\begin{equation}
\int^{+\delta t}_{-\delta t} d \Delta t\frac{1}{\sqrt{2\pi }\sigma}e^{-\Delta t^2 /2\sigma^2}=erf(\frac{\delta t}{\sqrt{2}\sigma})\approx\sqrt{\frac{2}{\pi}}\frac{\delta t}{\sigma}
\end{equation}
Where the last holds for the case $\delta t \ll \sigma$, so in this case
\begin{equation}
s =I_0B\sqrt{\frac{2}{\pi}}\frac{\delta t}{\sigma}
\end{equation}
While the noise is still given by Eq.(8), so in this case
\begin{equation}
{\rm SNR}_{fin} = \frac{I_0}{I_n}=\frac{I_0A_{\rm eff}}{2kT_{sys}}\left( \frac{BT}{N_t N_{\nu}} \right) ^{1/2} \sqrt{\frac{2}{\pi}}\frac{\delta t}{\sigma}
\end{equation}
\\ \\


\section{2D FFT dedispersion}
\label{sec:results}
The usual Fourier transform is :
\begin{equation}
\begin{aligned}
\widetilde{f}(\omega) & =\frac{1}{2\pi} \int f(t)e^{-i\omega t}dt \\
 f(t) & = \int \widetilde{f}(\omega)e^{i\omega t}d\omega
 \end{aligned}
\end{equation}
For$ f(t) = \delta _D(t-t_0), \widetilde{\omega} = \frac{1}{2\pi}e^{-i\omega t_0}$. Using the relation :
\begin{equation}
\int d\omega e^{i\omega t_0} = 2\pi \delta _D (t_0)
\end{equation}
we find the above indeed form a Fourier pair. However , here we want to use $\nu$ instead of $\omega$, then the Fourier transform pair are:
\begin{equation}
\begin{aligned}
\widetilde{f} (\nu) &= \frac{1}{2\pi}\int f(t) e^{-i2\pi \nu t}dt \\
f(t) &= 2\pi \int \widetilde{f}(\nu) e^{i2\pi \nu t} d\nu
\end{aligned}
\end{equation}
The 2d transform of the signal $I(\nu,t)$ is 
\begin{equation}
\widetilde{I}(f,\tau)=\int d\nu~e^{2\pi i \nu \tau} \int dt~e^{-2\pi ift}I(t,\nu)
\end{equation}
where we denote the Fourier conjugate variable of $\nu,t$ as $\tau,f$ to avoid confusion. For the pulse signal given by Eq.(2),
\begin{equation}
\begin{aligned}
\widetilde{I}(f,\tau) &= \int d\nu e^{2\pi i\nu \tau}I_0 e^{-2\pi i f(t_0 - \frac{2D}{\nu ^3 _0}\Delta \nu)} \\
								&= I_0 e^{-i2\pi f(t_0 + \frac{2D}{\nu ^2 _0})}\int d\nu~ e^{i2\pi \nu (\tau + \frac{2Df}{\nu ^2 _0})} \\
								&= \frac{I_0}{2\pi}exp[-i2\pi f(t_0 + \frac{2D}{\nu ^2 _0})]\delta
								 _D (\tau + \frac{2Df}{\nu ^2 _0})
\end{aligned}
\end{equation} 
Note $\widetilde{I}(\tau,f)$ is non-zero only on the staight line $\tau + \frac{2Df}{\nu ^2 _0}=0$, and the value is a complex number whose phase angle gives the arrival time. For the pulse with finite width,
\begin{equation}
\widetilde{I}(f,\tau) = \int d\nu e^{2\pi i \nu \tau}\int dt~e^{-2\pi ift} \frac{1}{\sqrt{2\pi}\sigma}e^{-\frac{(t-t')^2}{2\sigma ^2}}
\end{equation}
where $t'=t_0 + \frac{2D}{\nu _0 ^3}\Delta \nu$. Complete the t integral , we get
\begin{equation}
\begin{aligned}
\widetilde{I}(f,\tau)  & = \int d\nu~ e^{2\pi i\nu \tau}I_0  e^{-i2\pi ft'}e^{-\frac{(2\pi f\sigma)^2}{2}} \\
&= I_0e^{-i2\pi f(t_0 - \frac{2D}{\nu _0 ^2})}e^{-\frac{(2\pi f\sigma)^2}{2}}\int d\nu~exp[i2\pi\nu(\tau + \frac{2Df}{\nu ^2_0})] \\
&= \frac{I_0}{2\pi}e^{-i2\pi f(t_0-\frac{2D}{\nu ^2 _0})}e^{-\frac{(2\pi f\sigma)^2}{2}}\delta _D (\tau + \frac{2Df}{\nu ^2 _0})
\end{aligned}
\end{equation}
Note this is similar to Eq(18) except for the factor $e^{-\frac{(2\pi f \sigma)^2}{2}}$, this limits the usable range of f to $|f|<(2\pi\sigma)^{-1}$.
\\ \\
%%%%%%%%%%%%%%%%%%

\section{Transform to polar coordinates}
\label{sec:discussion}
We can take $\frac{2f}{\nu _0 ^2},\tau$ as the $x,y$ in Cartesian coordinates, then the polar cpprdinates $\rho,\theta$ can be defined as 
\begin{equation}
\begin{aligned}
\rho ^2 &= \left(\frac{2f}{\nu _0 ^2}\right) ^2 + \tau ^2 \\
\tan\theta &= \frac{\tau}{2f/\nu ^2 _0} 
\end{aligned}
\end{equation}
with $\tan \theta = -D$ for the track satisfy Eq.(20). Conversely, 
\begin{equation}
\begin{aligned}
f &= \frac{\nu ^2 _0}{2}\rho \cos\theta \\
\tau &= \rho \sin \theta
\end{aligned}
\end{equation}
Then 
\begin{equation}
\begin{aligned}
\widetilde{I}(\rho, \theta) = ~&\frac{I_0}{2\pi}e^{-i2\pi\left(\frac{\nu ^2 _0\cdot t}{2}-D \right)\rho\cos\theta}~\\ 
\cdot & ~e^{-\frac{\pi^2\sigma ^2 \nu _0 ^4}{2}\rho ^2 \cos ^2 \theta}\rho ^{-1} \delta _D (\theta + \arctan D)
\end{aligned}
\end{equation}
\subsection{Ionized Gas  Properties in the Host}

%The $\Halpha$ luminosity of the host also allows us to estimate the properties of the ionized ISM in the host and its contribution to the total DM of \frb.

The Balmer lines from the host also allow us to estimate the properties its ionized ISM  and its contribution to the total DM of \frb.

The $\Halpha$ surface density for the galaxy with  flux $F_{\halpha}$, semi-major axis $a$, and semi-minor axis $b$ is
\be
S(\halpha) &=& \frac{F_{\halpha}}{\pi a b}, \nonumber\\
&\approx&  6.8 \times 10^{-16}\,\mathrm{erg\,cm^{-2}\,s^{-1}\,arcsecond^{-2}}, \nonumber\\
&\approx&  120 \,\mathrm{Rayleigh},
\ee
where we have used the extinction corrected flux $F_{\halpha} = 2.6\times10^{-16}\,\mathrm{erg\,cm^{-2}\,s^{-1}}$ and the semi-major and minor axes ($a = 0\farcs44$, $b/a=0.68$) from the $i^\prime$ and $r^\prime$ images. In the source frame (denoted below by the subscript, `s'), the surface density is
% (1+z)^4 = (1 + 0.193)^4 = 2.03
\be
\Ss = (1+z)^4 S(\halpha) = 243\,\mathrm{Rayleigh}. 
\ee

For a temperature  $T = 10^4 T_4$~K, we express the  emission measure ($\EM\ = \int n_e^2 \mathrm{d}s$) given by \citet{reyn77}  in the galaxy's frame
\be
\EMsHalpha &=& 2.75 \mathrm{pc\,cm^{-6}} \, T_4^{0.9}\,  \left[\frac{\Ss}{\mathrm{Rayleigh}}\right], \nonumber \\
&\approx& 670\,\mathrm{pc\,cm^{-6}}\,T_4^{0.9}.
%\,(\theta_s/ 0\farcs7)^{-2}.
\ee
We get a smaller  value from the extinction-corrected  $\hbeta$ flux,  $\EMsHbeta \approx 530\,\mathrm{pc\,cm^{-6}}$. For the calculations below, we proceed with a combined  estimate,  $\EMs\ \approx 600\,\mathrm{pc\,cm^{-6}}$.

This value is fairly large compared to measurements of the local Galactic disk. The WHAM  $\Halpha$ survey, for example, gives values of tens of pc~cm$^{-6}$ in the Galactic plane  and about 1 pc~cm$^{-6}$ looking out of the plane \citep[][]{hbk+08}. However, 
lines of sight to distant pulsars and studies of other galaxies give EM values in the hundreds \citep[][]{reyn77, hdb+09}. 

The estimate for $\EMs$ is sensitive to the inferred solid angle of the galaxy and emitting regions. Ongoing observations with the \textit{Hubble Space Telescope} will better resolve the $\halpha$ emitting structures and improve our constraint on the EM with respect to the location of the burst. 

 The implied optical depth for free-free absorption at an observation frequency $\nu$ (in GHz) is
 \be
\tau_{\rm ff} &\approx&  3.3\times10^{-6} [(1+z)\nu_\mathrm{GHz}]^{-2.1} T_4^{-1.35} \EMs \nonumber \\ 
&\approx& 1.4 \times 10^{-3}  \nu_\mathrm{GHz}^{-2.1} T_4^{-0.45}.
\ee
%Therefore,
% unless a substantial fraction of the emitting gas is highly clumped, 
 Free-free absorption for \frb\ is therefore negligible even at 100\,MHz. This suggests that the radio spectra of the bursts and possibly the persistent source  are unaffected by absorption and are inherent to the emission process or to propagation effects near the sources, confirming the inference made by \citep{ssh+16b} based on the widely varying spectral shapes of the bursts alone.

 
 
\subsubsection{Implied DM from $\Halpha$-emitting Gas}
The EM implies a DM value sometimes given by $\DM = (\EM f_{\rm f} L)^{1/2}$, where
$f_{\rm f}$ is the volume filling factor of ionized clouds in a region of total size $L$ \citep{reyn77}. As summarized in Appendix~B of \citet[][]{cws+16},  additional fluctuations  decrease the  DM derived from EM, giving a source-frame value,
\be
\DMs &\approx & 
387\ \text{pc cm$^{-3}$}\, 
%550\ \text{pc cm$^{-3}$}\, 
	L_{\rm kpc}^{1/2}
	\left[ \frac{f_{\rm f}}{\zeta(1+\epsilon^2)/4}\right]^{1/2} \nonumber \\
& & \times \left(\frac{\EM}{600\ \text{pc cm$^{-6}$}} \right)^{1/2},
\label{eq:em2dm}
\ee
 where $\epsilon\le 1$ is the fractional variation inside discrete clouds due to turbulent-like density variations and  $\zeta\ge 1$ defines cloud-to-cloud density variations  in the ionized region of depth  $L_{\rm kpc}$  in kpc. Here we have used $\EMs\ = 600$ \text{pc cm$^{-6}$} and assumed 100\% cloud-to-cloud variations ($\zeta = 2$) and fully modulated electron densities inside clouds ($\epsilon=1$).



%We also note that since the galaxy is at $z=0.1927$, the effective contribution to the radio-measured DM (i.e. time delay as a function of frequency) will be a factor of $1/(1+z) = 0.84$ times the host galaxy DM, corresponding to $\DM(\Halpha) \lesssim230\,\mathrm{pc\,cm^{-3}}$. 

The host contribution to the {\it measured} \DM\ is a factor $(1+z)^{-1}$ smaller than the source frame DM\footnote{The factor of $(1+z)^{-1}$ is a combination of the photon redshift, time dilation and the frequency$^{-2}$ dependence of cold plasma dispersion.}.
%or $\DMhatzero = \DMs / (1+z) \approx 324 \ \text{pc cm$^{-3}$}$ multipled by the other factors in Eq.~\ref{eq:em2dm}.
Also, the line of sight to the  FRB source may  sample only a fraction of  $\DMs$  depending on if
it is embedded in or offset from the $\Halpha$-emitting gas.  For an effective path length through the ionized gas
$L_{\rm FRB} \le L$,  we then have
%\be
%\DMhatFRB & =& 
 %324 \ \text{pc cm$^{-3}$} \times
 %\nonumber \\
  %&&	\hspace{-0.75in}  
  %(L_{\rm FRB} / L)   (L_{\rm kpc}f_{\rm f})^{1/2}  \left[ 4/ \zeta(1+\epsilon^2)\right]^{1/2}.
%\ee
%
%{\bf Alternative to above equation: we could instead just say:}
\be
\DMhatFRB &=& \frac{\DMs}{1+z} \left(\frac{L_{\rm FRB}}{L} \right) 
\nonumber \\
&&\hspace{-0.85in}   \approx  324 \ \text{pc cm$^{-3}$}  \left(\frac{L_{\rm FRB}}{L} \right)
\left[\frac{4L_{\rm kpc}f_{\rm f}}{ \zeta(1+\epsilon^2)} \right]^{1/2}.
\ee

This estimate can be compared with empirical constraints  discussed in  \citet{clw+16} on contributions from the host and the intergalactic medium (IGM) to the total \DM\  made by subtracting the NE2001 model's DM contribution from the Milky Way \citep{cl02} %\authorcomment1{Jim/Joe: Pls add a comment on the uncertainties in MW+Halo contributions}  
($\DM_{\rm MW} = 188 \ \text{pc cm$^{-3}$}$) and the Milky Way halo ($\DM_{\rm MW_{halo}} = 30 \ \text{pc cm$^{-3}$}$) from the total $\DM\ = 558\,\mathrm{pc\,cm^{-3}}$. This gives $\DM_\mathrm{IGM} + \DM_\mathrm{host} = 340\,\mathrm{pc\,cm^{-3}}$.  
The Milky Way contributions have uncertain errors but are likely of order 20\%. 
The  measured redshift implies a mean IGM contribution $\DM_{\rm IGM} \approx 200\ \text{pc cm$^{-3}$}$ \citep{ioka03,inou04} but can vary by about $\pm 85 \ \text{pc cm$^{-3}$}$ \citep[][]{mcqu14}.   This yields a range of possible values for $\DM_{\rm host}$: $55 \lesssim \DM_{\rm host} \lesssim 225\,\mathrm{pc\,cm^{-3}}$ that further implies   
 %$0.12 \lesssim (L_{\rm FRB} / L)^{1/2}   f_{\rm f}^{1/2}  \left[ 4/ \zeta(1+\epsilon^2)\right]^{1/2} \lesssim 0.5$ or 
% $0.014 \lesssim (L_{\rm FRB} / L)   f_{\rm f}  \left[ 4/ \zeta(1+\epsilon^2)\right] \lesssim 0.25$.   
$0.09 \lesssim (L_{\rm FRB} / L) \left [L_{\rm kpc}  f_{\rm f} /  \zeta(1+\epsilon^2)\right]^{1/2} \lesssim 0.35$.
The ionized region therefore must have some degree of clumpiness or the effective path length is significantly  smaller than the size of the ionized region. 

%Radio pulsars believed to be in the Large Magellanic Cloud have DMs spanning the range 45--273\,$\mathrm{pc\,cm^{-3}}$; similarly Small Magellanic Cloud pulsars have DMs spanning the range 70--200 $\mathrm{pc\,cm^{-3}}$ \citep{mhth05}. 
Radio pulsars  in the Large and Small Magellanic Clouds have DMs spanning the range 45--273\,$\mathrm{pc\,cm^{-3}}$
 and  70--200 $\mathrm{pc\,cm^{-3}}$, respectively \citep{mhth05}. 
This empirically demonstrates that the free electron content of star-forming dwarf galaxies is of the order we estimate. The relatively large DM contribution from the host galaxy (as inferred from the $\Halpha$ emission) implies that any contributions from the vicinity of the FRB source itself are probably quite small. This may rule out a very young ($<100$\,yr) supernova remnant \citep[e.g. ][]{piro16}. 
%unless it was from a star stripped of its outer envelope before the explosion.



\subsection{Implications for Source Models}
\citet{clw+16} reported the locations of the radio bursts, the optical and variable radio counterparts and the absence of millimeter-wave and X-ray emission. \citet{mph+16} have shown that the bursts and the persistent radio source are colocated to within a linear projected separation of 40\,pc, suggesting that the two emission sources should be physically related, though not necessarily the same source. The radio source properties are consistent with a low luminosity AGN or a young ($<$1000\,yr) supernova remnant (SNR) powered by an energetic neutron star \citep[e.g. ][]{mkm16}. 

The optical properties of the galaxies reported here do not add support to the AGN interpretation although it cannot be conclusively ruled out. The BPT diagnostics for the host (Figure~\ref{fig:bpt}) show no indication of AGN activity. However, this may not be conclusive as the majority of radio-loud AGN show no optical signatures of activity \citep{ms07}. This is further supported by five low luminosity AGN with no optical signatures have also recently been discovered \citep{pyop16}. However, these objects are almost exclusively hosted in galaxies with much larger stellar masses ($\sim10^{10}\,M_\sun$).
We also note that the radio source is offset from the optical center of the galaxy by 170--300\,mas, corresponding to a transverse linear distance of 0.5--1\,kpc, nearly a quarter to half of the radial extent, which is not consistent with a central AGN, but such offsets have been seen before in dwarf galaxies, e.g. Henize 2-10 \citep{rsjb11}. 

The association of an optical/X-ray AGN with a dwarf galaxy is also extremely rare. A search of emission-line dwarf galaxies ($10^{8.5} \lesssim M_* \lesssim 10^{9.5}\,M_\sun$) using BPT line diagnostics identified an AGN rate of $\sim$ 0.5 \% \citep{rgg13}, with an additional 0.05 \% of dwarf galaxies searched exhibiting narrow emission lines consistent with star formation band broad \halpha\ consistent with an AGN. Similarly, an X-ray survey of $z<1$ dwarf galaxies reported an AGN rate of 0.6--3\% \citep{pgg+16}. Of the dwarf galaxies known to host AGN, only two exhibit nuclear radio emission that appears to originate from a black-hole jet, Henize 2-10 and Mrk 709 \citep{rsjb11,rpr+14}.  Both have strong nuclear X-ray emission that originates from the AGN but optical emission lines that are dominated by star-formation processes. The combination of a compact radio source, absent nuclear X-ray emission, strong star-formation optical emission lines, and weak or non-existent broad optical emission lines that we observe in the host of \frb\ has no analogue in any known galaxy to the best of our knowledge. 

The high star formation rate is consistent with the presence of a young SNR or a cluster of young massive stars (i.e. an OB association), which would naturally link FRBs to neutron stars which are the favored progenitor models.


\subsubsection{Relation to Dwarf Galaxies}
It is interesting to note that the only FRB host directly identified so far is a low metallicity dwarf galaxy rather than, say, an extremely high-star-formation-rate galaxy such as Arp\,220 or a galaxy with a very powerful AGN or some other extreme characteristics. Dwarf galaxies are also a small fraction of the stellar mass in the Universe \citep{pch+12}. \citet{rsb+16} also suggested that the extremely low scattering of FRB\,150807 compared to its DM may be linked to its origin from a low-mass ($<10^{9}\,M_\sun$) galaxy. However, the strong polarization and scattering properties of FRB\,110523 do suggest the presence of turbulent magnetized plasma around the source \citep{mls+15}, suggesting that individual FRB environments may be quite diverse.



%As dwarf galaxies are much more common than larger galaxies, while having lower total stellar mass and lower total star formation rate \citep{pch+12}, this discovery may point to an FRB source that is linked to the galactic host itself rather than its contents (\authorcomment1{I feel like we should take out this statement.}). Another way 

If FRBs are indeed more commonly hosted by dwarf galaxies in the low redshift Universe, they would share this preference with two other classes of high-energy transients --- long duration gamma-ray bursts and superluminous supernovae, both of which prefer low-mass, low-metallicity, and high star formation rate hosts \citep[e.g., ][ and other works]{fls+06,plt+13,vsj+15,pqy+16}. Indeed, superluminous supernovae are prefentially hosted by EELGs \citep{lsk+15}. If this relation is true, it may point to a link between FRBs and extremely massive progenitor stars, possibly extending to magnetars that have been associated with massive progenitor stars \citep[e.g.  ][]{ok14}.



\subsection{Future Optical Follow-Up of FRBs}
A link between FRBs and dwarf galaxies will impact future multi-wavelength follow-up plans. Without the precise localization for \frb\ \citep{clw+16}, the host galaxy is scarcely distinguishable from other objects in the deep Gemini images. 

 Due to the trade-off between field of view and localization precision, FRB search projects that have a large FRB detection rate such as CHIME (Kaspi V. M. et al,. 2017, in preparation), UTMOST \citep{cfb+16}, and HIRAX \citep{nbb+16} will localize high signal to noise detections to only sub-arcmin precision. If FRB hosts are star-forming galaxies with strong emission lines, slitless objective prism spectroscopy could efficiently distinguish these objects from a field of stars and elliptical galaxies, leading to putative host identifications without very precise localization. However, this strongly depends on the link between FRBs and their host properties and the homogeneity of FRBs --- which will first have to be confirmed with more interferometric localizations.

%We also note that \frb\ is in the Galactic plane ($b=-0.23\deg$) and hence has a very high density of objects and a large foreground extinction. Most FRBs detected at higher Galactic latitudes are in less crowded and less extincted fields and hence may not need as accurate a localization to identify a putative counterpart.

We note, of course, that our above discussion regarding the possible relationship between FRBs and dwarf galaxies in general is based on a single data point of a repeating FRB, which may not be representative of the broader FRB population \citep[see][for more details]{ssh+16a,ssh+16b}. 


%However, we do not yet know whether there are other galaxies around \frb's host with bright emission lines in the few arcsecond region that may lead to a false positive signal for such efforts --- that will be addressed in a future paper. 


%The largest uncertainty in this rate estimate is in the volume term, since we do not know the host DM contribution for other FRBs.
% To do:
% - Need to confirm that converting to "per galaxy" requires scaling by galaxy count of a given size.
% - Would be nice to make this independent of radio flux (standard candle?).


%\subsection{Summary}
%We have shown that the optical counterpart for \frb\ reported by \citet{clw+16} is a dwarf  galaxy ($M_*\sim4-7\times 10^{7}\,\mathrm{M_\sun}$) at a redshift of $z=0.1927$, consistent with the estimates from its DM. This proves the extragalactic nature of \frb, a first for any FRB. Thus we rule out any Galactic origin models for \frb\ and also show that the FRB source or its immediate environment need not separately contribute a significant DM to be viable, thus reducing the constraints for FRB source models. The host galaxy shows spectroscopic signatures of a high star formation rate ($0.4 \,\mathrm{M_\sun\,yr^{-1}}$) and no optical signatures of AGN activity. If FRBs are indeed generally related to dwarf galaxies, they may be linked to the two other phenomena preferring low-metallicity, low-mass hosts --- long duration $\gamma$-ray bursts and superluminous supernovae. Separately, if the FRB-dwarf galaxy relation holds, then identifying the optical counterparts of FRBs without very precise localization will be challenging due to their faintness and ubiquity of their potential hosts. 



\acknowledgements 
We are very grateful to the staff of the Gemini Observatory for their help and flexibility throughout this program. We also thank R.~F.~Trainor and A.~Delahaye for helpful discussions.

Our work is based on observations obtained at the Gemini Observatory (program GN-2016B-DD-2), which is operated by the Association of Universities for Research in Astronomy, Inc., under a cooperative agreement with the NSF on behalf of the Gemini partnership: the National Science Foundation (United States), the National Research Council (Canada), CONICYT (Chile), Ministerio de Ciencia, Tecnolog\'{i}a e Innovaci\'{o}n Productiva (Argentina), and Minist\'{e}rio da Ci\^{e}ncia, Tecnologia e Inova\c{c}\~{a}o (Brazil). 

This work has made use of data from the European Space Agency (ESA) mission {\it Gaia} (\url{http://www.cosmos.esa.int/gaia}), processed by the {\it Gaia} Data Processing and Analysis Consortium (DPAC, \url{http://www.cosmos.esa.int/web/gaia/dpac/consortium}). Funding for the DPAC has been provided by national institutions, in particular the institutions participating in the {\it Gaia} Multilateral Agreement. This research made use of Astropy, a community-developed core Python package for Astronomy (Astropy Collaboration, 2013, \url{http://www.astropy.org}).

S.P.T acknowledges support from a McGill Astrophysics postdoctoral fellowship. The research leading to these results has received funding from the European Research Council (ERC) under the European Union's Seventh Framework Programme (FP7/2007-2013). C.G.B. and J.W.T.H. gratefully acknowledge funding for this work from ERC Starting Grant DRAGNET under contract number 337062. J.M.C., R.S.W., and S.C. acknowledge prior support from the National Science Foundation through grants AST-1104617 and AST-1008213. This work was partially supported by the University of California Lab Fees program under award number LF-12-237863. The research leading to these results has received funding from the European Research Council (ERC) under the European Union’s Seventh Framework Programme (FP7/2007-2013).  J.W.T.H. is an NWO Vidi Fellow. V.M.K. holds the Lorne Trottier and a Canada Research Chair and receives support from an NSERC Discovery Grant and Accelerator Supplement, from a R. Howard Webster Foundation Fellowship from the Canadian Institute for Advanced Research (CIFAR), and from the FRQNT Centre de Recherche en Astrophysique du Quebec. B.M. acknowledges support by the Spanish Ministerio de Econom\'ia y Competitividad (MINECO/FEDER, UE) under grants AYA2013-47447-C3-1-P, AYA2016-76012-C3-1-P, and MDM-2014-0369 of ICCUB (Unidad de Excelencia `Mar\'ia de Maeztu'). L.G.S. gratefully acknowledge financial support from the ERC Starting Grant BEACON under contract number 279702 and the Max Planck Society. Part of this research was carried out at the Jet Propulsion Laboratory, California Institute of Technology, under a contract with the National Aeronautics and Space Administration. E.A.K.A. is supported by TOP1EW.14.105, which is financed by the Netherlands Organisation for Scientific Research (NWO). M.A.M. is supported by NSF award \#1458952. S.B.S is a Jansky Fellow of the National Radio Astronomy Observatory. P.S. is a Covington Fellow at the Dominion Radio Astrophysical Observatory.


\facility{Gemini:Gillett (GMOS)}
\software{ESO-MIDAS, astro-py, galfit, SExtractor}
%\bibliography{paper}

\begin{thebibliography}{}
\expandafter\ifx\csname natexlab\endcsname\relax\def\natexlab#1{#1}\fi
\providecommand{\url}[1]{\href{#1}{#1}}

\bibitem[{{Alam} {et~al.}(2015){Alam}, {Albareti}, {Allende Prieto}, {Anders},
  {Anderson}, {Anderton}, {Andrews}, {Armengaud}, {Aubourg}, {Bailey}, \&
  et~al.}]{aaa+15}
{Alam}, S., {Albareti}, F.~D., {Allende Prieto}, C., {et~al.} 2015, \apjs, 219,
  12

\bibitem[{{Astropy Collaboration} {et~al.}(2013){Astropy Collaboration},
  {Robitaille}, {Tollerud}, {Greenfield}, {Droettboom}, {Bray}, {Aldcroft},
  {Davis}, {Ginsburg}, {Price-Whelan}, {Kerzendorf}, {Conley}, {Crighton},
  {Barbary}, {Muna}, {Ferguson}, {Grollier}, {Parikh}, {Nair}, {Unther},
  {Deil}, {Woillez}, {Conseil}, {Kramer}, {Turner}, {Singer}, {Fox}, {Weaver},
  {Zabalza}, {Edwards}, {Azalee Bostroem}, {Burke}, {Casey}, {Crawford},
  {Dencheva}, {Ely}, {Jenness}, {Labrie}, {Lim}, {Pierfederici}, {Pontzen},
  {Ptak}, {Refsdal}, {Servillat}, \& {Streicher}}]{astropy2013}
{Astropy Collaboration}, {Robitaille}, T.~P., {Tollerud}, E.~J., {et~al.} 2013,
  \aap, 558, A33

\bibitem[{{Atek} {et~al.}(2011){Atek}, {Siana}, {Scarlata}, {Malkan},
  {McCarthy}, {Teplitz}, {Henry}, {Colbert}, {Bridge}, {Bunker}, {Dressler},
  {Fosbury}, {Hathi}, {Martin}, {Ross}, \& {Shim}}]{ass+11}
{Atek}, H., {Siana}, B., {Scarlata}, C., {et~al.} 2011, \apj, 743, 121

\bibitem[{{Baldwin} {et~al.}(1981){Baldwin}, {Phillips}, \&
  {Terlevich}}]{bpt81}
{Baldwin}, J.~A., {Phillips}, M.~M., \& {Terlevich}, R. 1981, \pasp, 93, 5

%\bibitem[{{Bandura} {et~al.}(2014){Bandura}, {Addison}, {Amiri}, {Bond},
%  {Campbell-Wilson}, {Connor}, {Cliche}, {Davis}, {Deng}, {Denman}, {Dobbs},
%  {Fandino}, {Gibbs}, {Gilbert}, {Halpern}, {Hanna}, {Hincks}, {Hinshaw},
%  {H{\"o}fer}, {Klages}, {Landecker}, {Masui}, {Mena Parra}, {Newburgh}, {Pen},
%  {Peterson}, {Recnik}, {Shaw}, {Sigurdson}, {Sitwell}, {Smecher}, {Smegal},
%  {Vanderlinde}, \& {Wiebe}}]{baa+14}
%{Bandura}, K., {Addison}, G.~E., {Amiri}, M., {et~al.} 2014, in \procspie, Vol.
%  9145, Ground-based and Airborne Telescopes V, 914522

\bibitem[{{Barentsen} {et~al.}(2014){Barentsen}, {Farnhill}, {Drew},
  {Gonz{\'a}lez-Solares}, {Greimel}, {Irwin}, {Miszalski}, {Ruhland}, {Groot},
  {Mampaso}, {Sale}, {Henden}, {Aungwerojwit}, {Barlow}, {Carter}, {Corradi},
  {Drake}, {Eisl{\"o}ffel}, {Fabregat}, {G{\"a}nsicke}, {Gentile Fusillo},
  {Greiss}, {Hales}, {Hodgkin}, {Huckvale}, {Irwin}, {King}, {Knigge},
  {Kupfer}, {Lagadec}, {Lennon}, {Lewis}, {Mohr-Smith}, {Morris}, {Naylor},
  {Parker}, {Phillipps}, {Pyrzas}, {Raddi}, {Roelofs}, {Rodr{\'{\i}}guez-Gil},
  {Sabin}, {Scaringi}, {Steeghs}, {Suso}, {Tata}, {Unruh}, {van Roestel},
  {Viironen}, {Vink}, {Walton}, {Wright}, \& {Zijlstra}}]{bfd+14}
{Barentsen}, G., {Farnhill}, H.~J., {Drew}, J.~E., {et~al.} 2014, \mnras, 444,
  3230

\bibitem[{{Bassa} {et~al.}(2016){Bassa}, {Beswick}, {Tingay}, {Keane},
  {Bhandari}, {Johnston}, {Totani}, {Tominaga}, {Yasuda}, {Stappers}, {Barr},
  {Kramer}, \& {Possenti}}]{bbt+16}
{Bassa}, C.~G., {Beswick}, R., {Tingay}, S.~J., {et~al.} 2016, \mnras, 463, L36

\bibitem[{{Bertin} \& {Arnouts}(1996)}]{ba96}
{Bertin}, E., \& {Arnouts}, S. 1996, \aaps, 117, 393

\bibitem[{{Burke-Spolaor} \& {Bannister}(2014)}]{bb14}
{Burke-Spolaor}, S., \& {Bannister}, K.~W. 2014, \apj, 792, 19

\bibitem[{{Caleb} {et~al.}(2016){Caleb}, {Flynn}, {Bailes}, {Barr}, {Bateman},
  {Bhandari}, {Campbell-Wilson}, {Green}, {Hunstead}, {Jameson}, {Jankowski},
  {Keane}, {Ravi}, {van Straten}, \& {Krishnan}}]{cfb+16}
{Caleb}, M., {Flynn}, C., {Bailes}, M., {et~al.} 2016, \mnras, 458, 718

\bibitem[{{Cardelli} {et~al.}(1989){Cardelli}, {Clayton}, \& {Mathis}}]{ccm89}
{Cardelli}, J.~A., {Clayton}, G.~C., \& {Mathis}, J.~S. 1989, \apj, 345, 245

\bibitem[{{Champion} {et~al.}(2016){Champion}, {Petroff}, {Kramer}, {Keith},
  {Bailes}, {Barr}, {Bates}, {Bhat}, {Burgay}, {Burke-Spolaor}, {Flynn},
  {Jameson}, {Johnston}, {Ng}, {Levin}, {Possenti}, {Stappers}, {van Straten},
  {Thornton}, {Tiburzi}, \& {Lyne}}]{cpk+16}
{Champion}, D.~J., {Petroff}, E., {Kramer}, M., {et~al.} 2016, \mnras, 460, L30

\bibitem[{{Chatterjee} {et~al.}(2017){Chatterjee}, {Law}, {Wharton},
  {Burke-Spolaor}, {Hessels}, {Bower}, {Cordes}, {Tendulkar}, {Bassa},
  {Demorest}, {Butler}, {Seymour}, {Scholz}, {Abruzzo}, {Bogdanov}, {Kaspi},
  {Keimpema}, {Lazio}, {Marcote}, {McLaughlin}, {Paragi}, {Ransom}, {Rupen},
  {Spitler}, \& {van Langevelde}}]{clw+16}
{Chatterjee}, S., {Law}, C.~J., {Wharton}, R.~S., {et~al.} 2017, \nat, 000, 000

\bibitem[{{Cordes} \& {Lazio}(2002)}]{cl02}
{Cordes}, J.~M., \& {Lazio}, T.~J.~W. 2002, ArXiv Astrophysics e-prints,
  astro-ph/0207156

\bibitem[{{Cordes} {et~al.}(2016){Cordes}, {Wharton}, {Spitler}, {Chatterjee},
  \& {Wasserman}}]{cws+16}
{Cordes}, J.~M., {Wharton}, R.~S., {Spitler}, L.~G., {Chatterjee}, S., \&
  {Wasserman}, I. 2016, ArXiv e-prints, arXiv:1605.05890

\bibitem[{{C{\^o}t{\'e}} {et~al.}(2000){C{\^o}t{\'e}}, {Carignan}, \&
  {Freeman}}]{ccf00}
{C{\^o}t{\'e}}, S., {Carignan}, C., \& {Freeman}, K.~C. 2000, \aj, 120, 3027

\bibitem[{{Dopita} {et~al.}(2016){Dopita}, {Kewley}, {Sutherland}, \&
  {Nicholls}}]{dksn16}
{Dopita}, M.~A., {Kewley}, L.~J., {Sutherland}, R.~S., \& {Nicholls}, D.~C.
  2016, \apss, 361, 61

\bibitem[{{Fruchter} {et~al.}(2006){Fruchter}, {Levan}, {Strolger},
  {Vreeswijk}, {Thorsett}, {Bersier}, {Burud}, {Castro Cer{\'o}n},
  {Castro-Tirado}, {Conselice}, {Dahlen}, {Ferguson}, {Fynbo}, {Garnavich},
  {Gibbons}, {Gorosabel}, {Gull}, {Hjorth}, {Holland}, {Kouveliotou}, {Levay},
  {Livio}, {Metzger}, {Nugent}, {Petro}, {Pian}, {Rhoads}, {Riess}, {Sahu},
  {Smette}, {Tanvir}, {Wijers}, \& {Woosley}}]{fls+06}
{Fruchter}, A.~S., {Levan}, A.~J., {Strolger}, L., {et~al.} 2006, \nat, 441,
  463

\bibitem[{{Gaia Collaboration} {et~al.}(2016){Gaia Collaboration}, {Brown},
  {Vallenari}, {Prusti}, {de Bruijne}, {Mignard}, {Drimmel}, \&
  {co-authors}}]{bvp+16}
{Gaia Collaboration}, {Brown}, A.~G.~A., {Vallenari}, A., {et~al.} 2016, ArXiv
  e-prints

\bibitem[{{Giroletti} {et~al.}(2016){Giroletti}, {Marcote}, {Garrett},
  {Paragi}, {Yang}, {Hada}, {Muxlow}, \& {Cheung}}]{gmg+16}
{Giroletti}, M., {Marcote}, B., {Garrett}, M.~A., {et~al.} 2016, \aap, 593, L16

\bibitem[{{Graham} \& {Fruchter}(2015)}]{gf15b}
{Graham}, J.~F., \& {Fruchter}, A.~S. 2015, ArXiv e-prints, arXiv:1511.01079

\bibitem[{{Haffner} {et~al.}(2009){Haffner}, {Dettmar}, {Beckman}, {Wood},
  {Slavin}, {Giammanco}, {Madsen}, {Zurita}, \& {Reynolds}}]{hdb+09}
{Haffner}, L.~M., {Dettmar}, R.-J., {Beckman}, J.~E., {et~al.} 2009, Reviews of
  Modern Physics, 81, 969

\bibitem[{{Hamuy} {et~al.}(1994){Hamuy}, {Suntzeff}, {Heathcote}, {Walker},
  {Gigoux}, \& {Phillips}}]{hsh+94}
{Hamuy}, M., {Suntzeff}, N.~B., {Heathcote}, S.~R., {et~al.} 1994, \pasp, 106,
  566

\bibitem[{{Hamuy} {et~al.}(1992){Hamuy}, {Walker}, {Suntzeff}, {Gigoux},
  {Heathcote}, \& {Phillips}}]{hws+92}
{Hamuy}, M., {Walker}, A.~R., {Suntzeff}, N.~B., {et~al.} 1992, \pasp, 104, 533

\bibitem[{{Hill} {et~al.}(2008){Hill}, {Benjamin}, {Kowal}, {Reynolds},
  {Haffner}, \& {Lazarian}}]{hbk+08}
{Hill}, A.~S., {Benjamin}, R.~A., {Kowal}, G., {et~al.} 2008, \apj, 686, 363

\bibitem[{{Horne}(1986)}]{hor86}
{Horne}, K. 1986, \pasp, 98, 609

\bibitem[{{Hynes}(2002)}]{hyn02}
{Hynes}, R.~I. 2002, \aap, 382, 752

\bibitem[{{Inoue}(2004)}]{inou04}
{Inoue}, S. 2004, \mnras, 348, 999

\bibitem[{{Ioka}(2003)}]{ioka03}
{Ioka}, K. 2003, \apjl, 598, L79

\bibitem[{{Johnston} {et~al.}(2017){Johnston}, {Keane}, {Bhandari}, {Macquart},
  {Tingay}, {Barr}, {Bassa}, {Beswick}, {Burgay}, {Chandra}, {Honma}, {Kramer},
  {Petroff}, {Possenti}, {Stappers}, \& {Sugai}}]{jkb+16}
{Johnston}, S., {Keane}, E.~F., {Bhandari}, S., {et~al.} 2017, \mnras, 465,
  2143

\bibitem[{{Katz}(2016)}]{katz16}
{Katz}, J.~I. 2016, Modern Physics Letters A, 31, 1630013

\bibitem[{{Kauffmann} {et~al.}(2003){Kauffmann}, {Heckman}, {Tremonti},
  {Brinchmann}, {Charlot}, {White}, {Ridgway}, {Brinkmann}, {Fukugita}, {Hall},
  {Ivezi{\'c}}, {Richards}, \& {Schneider}}]{kht+03}
{Kauffmann}, G., {Heckman}, T.~M., {Tremonti}, C., {et~al.} 2003, \mnras, 346,
  1055

\bibitem[{{Keane} {et~al.}(2012){Keane}, {Stappers}, {Kramer}, \&
  {Lyne}}]{kskl12}
{Keane}, E.~F., {Stappers}, B.~W., {Kramer}, M., \& {Lyne}, A.~G. 2012, \mnras,
  425, L71

\bibitem[{{Keane} {et~al.}(2016){Keane}, {Johnston}, {Bhandari}, {Barr},
  {Bhat}, {Burgay}, {Caleb}, {Flynn}, {Jameson}, {Kramer}, {Petroff},
  {Possenti}, {van Straten}, {Bailes}, {Burke-Spolaor}, {Eatough}, {Stappers},
  {Totani}, {Honma}, {Furusawa}, {Hattori}, {Morokuma}, {Niino}, {Sugai},
  {Terai}, {Tominaga}, {Yamasaki}, {Yasuda}, {Allen}, {Cooke}, {Jencson},
  {Kasliwal}, {Kaplan}, {Tingay}, {Williams}, {Wayth}, {Chandra}, {Perrodin},
  {Berezina}, {Mickaliger}, \& {Bassa}}]{kjb+16}
{Keane}, E.~F., {Johnston}, S., {Bhandari}, S., {et~al.} 2016, \nat, 530, 453

\bibitem[{{Kennicutt} {et~al.}(1994){Kennicutt}, {Tamblyn}, \&
  {Congdon}}]{ktc1994}
{Kennicutt}, Jr., R.~C., {Tamblyn}, P., \& {Congdon}, C.~E. 1994, \apj, 435, 22

\bibitem[{{Kewley} \& {Dopita}(2002)}]{kd02}
{Kewley}, L.~J., \& {Dopita}, M.~A. 2002, \apjs, 142, 35

\bibitem[{{Kewley} {et~al.}(2001){Kewley}, {Dopita}, {Sutherland}, {Heisler},
  \& {Trevena}}]{kds+01}
{Kewley}, L.~J., {Dopita}, M.~A., {Sutherland}, R.~S., {Heisler}, C.~A., \&
  {Trevena}, J. 2001, \apj, 556, 121

\bibitem[{{Kewley} \& {Ellison}(2008)}]{ke08}
{Kewley}, L.~J., \& {Ellison}, S.~L. 2008, \apj, 681, 1183

\bibitem[{{Kewley} {et~al.}(2002){Kewley}, {Geller}, {Jansen}, \&
  {Dopita}}]{kgjd2002}
{Kewley}, L.~J., {Geller}, M.~J., {Jansen}, R.~A., \& {Dopita}, M.~A. 2002,
  \aj, 124, 3135

\bibitem[{{Kewley} {et~al.}(2006){Kewley}, {Groves}, {Kauffmann}, \&
  {Heckman}}]{kgkh06}
{Kewley}, L.~J., {Groves}, B., {Kauffmann}, G., \& {Heckman}, T. 2006, \mnras,
  372, 961

\bibitem[{{Kobulnicky} \& {Kewley}(2004)}]{kk04}
{Kobulnicky}, H.~A., \& {Kewley}, L.~J. 2004, \apj, 617, 240

\bibitem[{{Lelli} {et~al.}(2014){Lelli}, {Verheijen}, \& {Fraternali}}]{lvf14}
{Lelli}, F., {Verheijen}, M., \& {Fraternali}, F. 2014, \aap, 566, A71

\bibitem[{{Leloudas} {et~al.}(2015){Leloudas}, {Schulze}, {Kr{\"u}hler},
  {Gorosabel}, {Christensen}, {Mehner}, {de Ugarte Postigo}, {Amor{\'{\i}}n},
  {Th{\"o}ne}, {Anderson}, {Bauer}, {Gallazzi}, {He{\l}miniak}, {Hjorth},
  {Ibar}, {Malesani}, {Morell}, {Vinko}, \& {Wheeler}}]{lsk+15}
{Leloudas}, G., {Schulze}, S., {Kr{\"u}hler}, T., {et~al.} 2015, \mnras, 449,
  917

\bibitem[{{Lorimer} {et~al.}(2007){Lorimer}, {Bailes}, {McLaughlin},
  {Narkevic}, \& {Crawford}}]{lbm+07}
{Lorimer}, D.~R., {Bailes}, M., {McLaughlin}, M.~A., {Narkevic}, D.~J., \&
  {Crawford}, F. 2007, Science, 318, 777

\bibitem[{{Manchester} {et~al.}(2005){Manchester}, {Hobbs}, {Teoh}, \&
  {Hobbs}}]{mhth05}
{Manchester}, R.~N., {Hobbs}, G.~B., {Teoh}, A., \& {Hobbs}, M. 2005, \aj, 129,
  1993

\bibitem[{{Marcote} {et~al.}(2017){Marcote}, {Paragi}, {Hessels}, {Keimpema},
  \& {van Langevelde}}]{mph+16}
{Marcote}, B., {Paragi}, Z., {Hessels}, J.~W.~T., {Keimpema}, A., \& {van
  Langevelde}, H.~J. e.~a. 2017, \apj, 000, 000

\bibitem[{{Masui} {et~al.}(2015){Masui}, {Lin}, {Sievers}, {Anderson}, {Chang},
  {Chen}, {Ganguly}, {Jarvis}, {Kuo}, {Li}, {Liao}, {McLaughlin}, {Pen},
  {Peterson}, {Roman}, {Timbie}, {Voytek}, \& {Yadav}}]{mls+15}
{Masui}, K., {Lin}, H.-H., {Sievers}, J., {et~al.} 2015, \nat, 528, 523

\bibitem[{{Mauch} \& {Sadler}(2007)}]{ms07}
{Mauch}, T., \& {Sadler}, E.~M. 2007, \mnras, 375, 931

\bibitem[{{McQuinn}(2014)}]{mcqu14}
{McQuinn}, M. 2014, \apjl, 780, L33

\bibitem[{{Mignard} {et~al.}(2016){Mignard}, {Klioner}, {Lindegren}, {Bastian},
  {Bombrun}, {Hernandez}, {Hobbs}, {Lammers}, {Michalik}, {Ramos-Lerate},
  {Biermann}, {Butkevich}, {Comoretto}, {Joliet}, {Holl}, {Hutton}, {Parsons},
  {Steidelmueller}, {Andrei}, {Bourda}, \& {Charlot}}]{mkl+16}
{Mignard}, F., {Klioner}, S., {Lindegren}, L., {et~al.} 2016, ArXiv e-prints,
  arXiv:1609.07255

\bibitem[{{Moffat}(1969)}]{mof69}
{Moffat}, A.~F.~J. 1969, \aap, 3, 455

\bibitem[{{Murase} {et~al.}(2016){Murase}, {Kashiyama}, \&
  {M{\'e}sz{\'a}ros}}]{mkm16}
{Murase}, K., {Kashiyama}, K., \& {M{\'e}sz{\'a}ros}, P. 2016, \mnras, 461,
  1498

\bibitem[{{Newburgh} {et~al.}(2016){Newburgh}, {Bandura}, {Bucher}, {Chang},
  {Chiang}, {Cliche}, {Dav{\'e}}, {Dobbs}, {Clarkson}, {Ganga}, {Gogo},
  {Gumba}, {Gupta}, {Hilton}, {Johnstone}, {Karastergiou}, {Kunz}, {Lokhorst},
  {Maartens}, {Macpherson}, {Mdlalose}, {Moodley}, {Ngwenya}, {Parra},
  {Peterson}, {Recnik}, {Saliwanchik}, {Santos}, {Sievers}, {Smirnov},
  {Stronkhorst}, {Taylor}, {Vanderlinde}, {Van Vuuren}, {Weltman}, \&
  {Witzemann}}]{nbb+16}
{Newburgh}, L.~B., {Bandura}, K., {Bucher}, M.~A., {et~al.} 2016, in \procspie,
  Vol. 9906, Society of Photo-Optical Instrumentation Engineers (SPIE)
  Conference Series, 99065X

\bibitem[{{Olausen} \& {Kaspi}(2014)}]{ok14}
{Olausen}, S.~A., \& {Kaspi}, V.~M. 2014, \apjs, 212, 6

\bibitem[{{Papastergis} {et~al.}(2012){Papastergis}, {Cattaneo}, {Huang},
  {Giovanelli}, \& {Haynes}}]{pch+12}
{Papastergis}, E., {Cattaneo}, A., {Huang}, S., {Giovanelli}, R., \& {Haynes},
  M.~P. 2012, \apj, 759, 138

\bibitem[{{Pardo} {et~al.}(2016){Pardo}, {Goulding}, {Greene}, {Somerville},
  {Gallo}, {Hickox}, {Miller}, {Reines}, \& {Silverman}}]{pgg+16}
{Pardo}, K., {Goulding}, A.~D., {Greene}, J.~E., {et~al.} 2016, \apj, 831, 203

\bibitem[{{Park} {et~al.}(2016){Park}, {Yang}, {Oonk}, \& {Paragi}}]{pyop16}
{Park}, S., {Yang}, J., {Oonk}, J.~B.~R., \& {Paragi}, Z. 2016, ArXiv e-prints,
  arXiv:1611.05986

\bibitem[{{Peng} {et~al.}(2002){Peng}, {Ho}, {Impey}, \& {Rix}}]{phir02}
{Peng}, C.~Y., {Ho}, L.~C., {Impey}, C.~D., \& {Rix}, H.-W. 2002, \aj, 124, 266

\bibitem[{{Peng} {et~al.}(2010){Peng}, {Ho}, {Impey}, \& {Rix}}]{phir10}
---. 2010, \aj, 139, 2097

\bibitem[{{Perley} {et~al.}(2013){Perley}, {Levan}, {Tanvir}, {Cenko}, {Bloom},
  {Hjorth}, {Kr{\"u}hler}, {Filippenko}, {Fruchter}, {Fynbo}, {Jakobsson},
  {Kalirai}, {Milvang-Jensen}, {Morgan}, {Prochaska}, \& {Silverman}}]{plt+13}
{Perley}, D.~A., {Levan}, A.~J., {Tanvir}, N.~R., {et~al.} 2013, \apj, 778, 128

\bibitem[{{Perley} {et~al.}(2016){Perley}, {Quimby}, {Yan}, {Vreeswijk}, {De
  Cia}, {Lunnan}, {Gal-Yam}, {Yaron}, {Filippenko}, {Graham}, {Laher}, \&
  {Nugent}}]{pqy+16}
{Perley}, D.~A., {Quimby}, R.~M., {Yan}, L., {et~al.} 2016, \apj, 830, 13

\bibitem[{{Petroff} {et~al.}(2015){Petroff}, {Bailes}, {Barr}, {Barsdell},
  {Bhat}, {Bian}, {Burke-Spolaor}, {Caleb}, {Champion}, {Chandra}, {Da Costa},
  {Delvaux}, {Flynn}, {Gehrels}, {Greiner}, {Jameson}, {Johnston}, {Kasliwal},
  {Keane}, {Keller}, {Kocz}, {Kramer}, {Leloudas}, {Malesani}, {Mulchaey},
  {Ng}, {Ofek}, {Perley}, {Possenti}, {Schmidt}, {Shen}, {Stappers},
  {Tisserand}, {van Straten}, \& {Wolf}}]{pbb+15}
{Petroff}, E., {Bailes}, M., {Barr}, E.~D., {et~al.} 2015, \mnras, 447, 246

\bibitem[{{Pettini} \& {Pagel}(2004)}]{pp04}
{Pettini}, M., \& {Pagel}, B.~E.~J. 2004, \mnras, 348, L59

\bibitem[{{Piro}(2016)}]{piro16}
{Piro}, A.~L. 2016, \apjl, 824, L32

\bibitem[{{Planck Collaboration} {et~al.}(2016){Planck Collaboration}, {Ade},
  {Aghanim}, {Arnaud}, {Ashdown}, {Aumont}, {Baccigalupi}, {Banday},
  {Barreiro}, {Bartlett}, \& et~al.}]{aaa+16}
{Planck Collaboration}, {Ade}, P.~A.~R., {Aghanim}, N., {et~al.} 2016, \aap,
  594, A13

\bibitem[{{Ravi} {et~al.}(2015){Ravi}, {Shannon}, \& {Jameson}}]{rsj15}
{Ravi}, V., {Shannon}, R.~M., \& {Jameson}, A. 2015, \apjl, 799, L5

\bibitem[{{Ravi} {et~al.}(2016){Ravi}, {Shannon}, {Bailes}, {Bannister},
  {Bhandari}, {Bhat}, {Burke-Spolaor}, {Caleb}, {Flynn}, {Jameson}, {Johnston},
  {Keane}, {Kerr}, {Tiburzi}, {Tuntsov}, \& {Vedantham}}]{rsb+16}
{Ravi}, V., {Shannon}, R.~M., {Bailes}, M., {et~al.} 2016, ArXiv e-prints,
  arXiv:1611.05758

\bibitem[{{Reines} {et~al.}(2013){Reines}, {Greene}, \& {Geha}}]{rgg13}
{Reines}, A.~E., {Greene}, J.~E., \& {Geha}, M. 2013, \apj, 775, 116

\bibitem[{{Reines} {et~al.}(2014){Reines}, {Plotkin}, {Russell}, {Mezcua},
  {Condon}, {Sivakoff}, \& {Johnson}}]{rpr+14}
{Reines}, A.~E., {Plotkin}, R.~M., {Russell}, T.~D., {et~al.} 2014, \apjl, 787,
  L30

\bibitem[{{Reines} {et~al.}(2011){Reines}, {Sivakoff}, {Johnson}, \&
  {Brogan}}]{rsjb11}
{Reines}, A.~E., {Sivakoff}, G.~R., {Johnson}, K.~E., \& {Brogan}, C.~L. 2011,
  \nat, 470, 66

\bibitem[{{Reynolds}(1977)}]{reyn77}
{Reynolds}, R.~J. 1977, \apj, 216, 433

\bibitem[{{Schlafly} \& {Finkbeiner}(2011)}]{sf11}
{Schlafly}, E.~F., \& {Finkbeiner}, D.~P. 2011, \apj, 737, 103

\bibitem[{{Schlafly} {et~al.}(2010){Schlafly}, {Finkbeiner}, {Schlegel},
  {Juri{\'c}}, {Ivezi{\'c}}, {Gibson}, {Knapp}, \& {Weaver}}]{sfs10}
{Schlafly}, E.~F., {Finkbeiner}, D.~P., {Schlegel}, D.~J., {et~al.} 2010, \apj,
  725, 1175

\bibitem[{{Schlegel} {et~al.}(1998){Schlegel}, {Finkbeiner}, \&
  {Davis}}]{sfd98}
{Schlegel}, D.~J., {Finkbeiner}, D.~P., \& {Davis}, M. 1998, \apj, 500, 525

\bibitem[{{Scholz} {et~al.}(2016){Scholz}, {Spitler}, {Hessels}, {Chatterjee},
  {Cordes}, {Kaspi}, {Wharton}, {Bassa}, {Bogdanov}, {Camilo}, {Crawford},
  {Deneva}, {van Leeuwen}, {Lynch}, {Madsen}, {McLaughlin}, {Mickaliger},
  {Parent}, {Patel}, {Ransom}, {Seymour}, {Stairs}, {Stappers}, \&
  {Tendulkar}}]{ssh+16b}
{Scholz}, P., {Spitler}, L.~G., {Hessels}, J.~W.~T., {et~al.} 2016, \apj, 883,
  177

\bibitem[{{Spitler} {et~al.}(2014){Spitler}, {Cordes}, {Hessels}, {Lorimer},
  {McLaughlin}, {Chatterjee}, {Crawford}, {Deneva}, {Kaspi}, {Wharton},
  {Allen}, {Bogdanov}, {Brazier}, {Camilo}, {Freire}, {Jenet},
  {Karako-Argaman}, {Knispel}, {Lazarus}, {Lee}, {van Leeuwen}, {Lynch},
  {Ransom}, {Scholz}, {Siemens}, {Stairs}, {Stovall}, {Swiggum},
  {Venkataraman}, {Zhu}, {Aulbert}, \& {Fehrmann}}]{sch+14}
{Spitler}, L.~G., {Cordes}, J.~M., {Hessels}, J.~W.~T., {et~al.} 2014, \apj,
  790, 101

\bibitem[{{Spitler} {et~al.}(2016){Spitler}, {Scholz}, {Hessels}, {Bogdanov},
  {Brazier}, {Camilo}, {Chatterjee}, {Cordes}, {Crawford}, {Deneva}, {Ferdman},
  {Freire}, {Kaspi}, {Lazarus}, {Lynch}, {Madsen}, {McLaughlin}, {Patel},
  {Ransom}, {Seymour}, {Stairs}, {Stappers}, {van Leeuwen}, \& {Zhu}}]{ssh+16a}
{Spitler}, L.~G., {Scholz}, P., {Hessels}, J.~W.~T., {et~al.} 2016, \nat, 531,
  202

\bibitem[{{Thornton} {et~al.}(2013){Thornton}, {Stappers}, {Bailes},
  {Barsdell}, {Bates}, {Bhat}, {Burgay}, {Burke-Spolaor}, {Champion}, {Coster},
  {D'Amico}, {Jameson}, {Johnston}, {Keith}, {Kramer}, {Levin}, {Milia}, {Ng},
  {Possenti}, \& {van Straten}}]{tsb+13}
{Thornton}, D., {Stappers}, B., {Bailes}, M., {et~al.} 2013, Science, 341, 53

\bibitem[{{Vergani} {et~al.}(2015){Vergani}, {Salvaterra}, {Japelj}, {Le
  Floc'h}, {D'Avanzo}, {Fernandez-Soto}, {Kr{\"u}hler}, {Melandri}, {Boissier},
  {Covino}, {Puech}, {Greiner}, {Hunt}, {Perley}, {Petitjean}, {Vinci},
  {Hammer}, {Levan}, {Mannucci}, {Campana}, {Flores}, {Gomboc}, \&
  {Tagliaferri}}]{vsj+15}
{Vergani}, S.~D., {Salvaterra}, R., {Japelj}, J., {et~al.} 2015, \aap, 581,
  A102

\bibitem[{{Williams} \& {Berger}(2016)}]{wb16}
{Williams}, P.~K.~G., \& {Berger}, E. 2016, \apjl, 821, L22

\end{thebibliography}




\end{document}



%\begin{figure*}
%  \center
%  \includegraphics[width=0.9\textwidth]{nustar_regions}
%  \includegraphics[clip=true,trim=0.3in 0.3in 1.1in 0.8in,width=0.45\textwidth]{backgrounds_A}
%  \includegraphics[clip=true,trim=0.3in 0.3in 1.1in 0.8in,width=0.45\textwidth]{backgrounds_B}
%  \caption{}
%  \label{fig:nustar_regions}
%\end{figure*}


%\begin{deluxetable}{lcccc}
%  \centering
%  \tablecolumns{5} 
%  \tablecaption{X-ray Observations of \igr\ in 2014.\label{tab:obs}}
%  \tablewidth{0pt}
%  \tabletypesize{\footnotesize}
%  \tablehead{
%    \colhead{Obs ID/Rev\tablenotemark{a}}   &
%    \multicolumn{2}{c}{Obs Time (UT)\tablenotemark{b}} & 
%    \colhead{Exp}      &
%    \colhead{Rate}\\
%    \colhead{}  &
%    \colhead{Start}                    &
%    \colhead{End}                    &
%    \colhead{(ks)}                 &
%    \colhead{(cts/s)}
%  }
%  \startdata 
%  \sidehead{\nustar}           %MJD-OBS            TELAPSE  
%  80001046002 & 10-10 08:51:07 & 10-11 00:46:07 & 29.4& 3.5--4  \\  %56940.38413696764  5.562777499043941E+04

%  \sidehead{\integral}
%  1458        & 09-21 19:03:49 & 09-24 04:25:41 & 28.7  & 3.1\\  

% \enddata
%\tablecomments{}
%\tablenotetext{a}{}
%\tablenotetext{b}{}
%\tablenotetext{p}{}
%\end{deluxetable}

%\begin{deluxetable}{clcc}
%  \centering
%  \tablecolumns{4} 
%  \tablecaption{Optical Observations of \frb \label{tab:obs}}
%  \tablewidth{0pt}
%  \tabletypesize{\footnotesize}
%  \tablehead{
%    % Date, Filt/Grating, Exposure, Conditions 
%    \colhead{Date}   &
%    \colhead{Filter/} & 
%    \colhead{Exposure}      &
%    \colhead{Conditions}\\
%    \colhead{UTC}  &
%    \colhead{Grating}  &
%    \colhead{(s)} &
%    \colhead{\texttt{IQ},\texttt{CC},\texttt{WV},\texttt{BG}\tablenotemark{a}}                 
%  }
%  \startdata 
% % \sidehead{\nustar}           %MJD-OBS            TELAPSE  
% % 80001046002 & 10-10 08:51:07 & 10-11 00:46:07 & 29.4& 3.5--4  \\  %56940.38413696764  5.562777499043941E+04
%2016-10-24 & $r^\prime$ & 500 & 20,70,50,50\\
%2016-10-25 & $r^\prime$ & 750 & 70,70,20,20\\
%2016-11-02 & $i^\prime$ & 1200 & 20,50,20,20\\
%2016-11-02 & $z^\prime$ & 1800 & 70,50,20,20\\
%2016-11-09 & R400/675\tablenotemark{b} & 7200 & 20,50,100,20\\
%2016-11-10 & R400/690\tablenotemark{b} & 5400 & 70,50,100,20\\
%2016-11-10 & R400/675\tablenotemark{b} & 3600 & 70,50,100,20\\
% \enddata
%%\tablecomments{}
%\tablenotetext{a}{Image quality (\texttt{IQ}), Cloud Cover (\texttt{CC}), Water Vapor (\texttt{WV}), and Sky Background (\texttt{BG}) conditions in percentiles for the Gemini North Observatory site on Mauna Kea. \texttt{IQ}20 and \texttt{IQ}70 correspond to 0\farcs50 and 0\farcs75 zenith seeing in $i^\prime$ band, respectively. \texttt{BG}20 and \texttt{BG}50 correspond to sky surface brightness of $\upmu_V > 21.3\,\mathrm{mag\,arcsec^{-2}}$ and $ > 20.7\,\mathrm{mag\,arcsec^{-2}}$, respectively. }
%\tablenotetext{b}{Grating and central wavelength in nm.}
%\end{deluxetable}


